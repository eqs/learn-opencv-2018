\documentclass[10pt]{beamer}

\usefonttheme[onlymath]{serif}

\usepackage{amsmath,amssymb}
\usepackage{zxjatype}
\usepackage[ipa]{zxjafont}
\usetheme{metropolis}
\usepackage{tikz}
\usepackage{tikzsymbols}
\usepackage{appendixnumberbeamer}
\usepackage{minted}

% --- page number ---
\setbeamertemplate{footline}{%
	\raisebox{10pt}{\makebox[\paperwidth]{\hfill\makebox[7em]{\normalsize\texttt{\insertframenumber/\inserttotalframenumber}}}}%
}

% --- title logo ---
\newcommand{\myinsertlogo}[1]{%
\begin{tikzpicture}[overlay, remember picture]
    \node[above left=1cm and .8cm of current page.south east] {\includegraphics[width=2.25cm]{#1}};
\end{tikzpicture}}

% --- utils ---
\newcommand{\mymain}[1]{\textcolor{mLightBrown}{#1}}
\newcommand{\myaccent}[1]{\textcolor{mLightGreen}{#1}}

\title{Introduction to OpenCV}
\date{\today}
\author{Satoshi Murashige}
\institute{Mathematical Informatics Lab., NAIST}

\begin{document}
	\begin{frame}[plain]
		\maketitle
		%\myinsertlogo{spam.pdf}
	\end{frame}
	\begin{frame}{Text books}
		『詳解 OpenCV』
	\end{frame}
	\begin{frame}{Table of Contents}
		\begin{enumerate}
			\item 基本的な画像処理
				\begin{itemize}
					\item 1章:概要
					\item 2章:OpenCV入門
					\item 10章:フィルタとコンボリューション
				\end{itemize}
			\item 物体検出
				\begin{itemize}
					\item 12章:画像解析
					\item 13章:ヒストグラムとテンプレートマッチング
					\item 14章:輪郭
				\end{itemize}
			\item 動画解析
				\begin{itemize}
					\item 15章:背景除去
					\item 16章:キーポイントと記述子
					\item 17章:トラッキング
				\end{itemize}
			\item 3次元復元
				\begin{itemize}
					\item 18章:カメラモデルとキャリブレーション
					\item 19章:射影変換と3次元ビジョン
				\end{itemize}
		\end{enumerate}
	\end{frame}

	\section{1章:概要}

	\begin{frame}{title}
		aaa
	\end{frame}
	
	\section{2章:OpenCV入門}
	
	\begin{frame}{title}
		aaa
	\end{frame}
	
	\section{10章:フィルタとコンボリューション}
	
	\begin{frame}{画像フィルタリング}
		\begin{itemize}
			\item 画像を色の値からなる「2次元配列」ではなく,「2変数関数」と解釈する
			\item 画像フィルタリング:入力画像$I(x, y)$から新しい画像$I'(x, y)$を計算するアルゴリズム
				\begin{itemize}
					\item 例1:ある画像からぼけた画像を生成する
					\item 例2:ある画像を白と黒のみからなる画像に変換する
				\end{itemize}
		\end{itemize}
	\end{frame}
	
	\begin{frame}{画像フィルタリングの内容はカーネルによって定義される}
		\begin{itemize}
			\item 出力画像$I'(x, y)$の位置$(x, y)$における画素値は入力画像中の位置$(x, y)$周辺の画素から計算される
			\[ I'(x, y) = \sum_{i, j \in \mathit{kernel}} k_{i, j} \cdot I(x + i, y + j) \]
			\item 上式中の$k_{i, j}$を\mymain{線形カーネル(フィルタ)}と呼ぶ
			\item 画像に対してカーネル(線形,非線形問わず)を
				適用する操作を\mymain{コンボリューション}と呼ぶ
		\end{itemize}
	\end{frame}

	\begin{frame}{線形カーネルと適用のイメージ}
		\begin{center}
			\begin{tabular}{|c|c|c|}\hline
				$\frac{1}{9}$ & $\frac{1}{9}$ & $\frac{1}{9}$ \\ \hline
				$\frac{1}{9}$ & $\frac{1}{9}$ & $\frac{1}{9}$ \\ \hline
				$\frac{1}{9}$ & $\frac{1}{9}$ & $\frac{1}{9}$ \\ \hline
			\end{tabular}
		\end{center}
	\end{frame}

	\begin{frame}[fragile]{線形カーネルによる画像フィルタリングの例:平滑化}

	\end{frame}
	
	\begin{frame}{境界の設定}
		
	\end{frame}

\end{document}
