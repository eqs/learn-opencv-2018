\documentclass[10pt]{beamer}

\usepackage{amsmath,amssymb}
\usepackage{zxjatype}
\usepackage[ipa]{zxjafont}
\usetheme{metropolis}
\usepackage{tikz}
\usepackage{tikzsymbols}
\usepackage{appendixnumberbeamer}

% --- page number ---
\setbeamertemplate{footline}{%
	\raisebox{10pt}{\makebox[\paperwidth]{\hfill\makebox[7em]{\normalsize\texttt{\insertframenumber/\inserttotalframenumber}}}}%
}

% --- title logo ---

\newcommand{\myinsertlogo}[1]{%
\begin{tikzpicture}[overlay, remember picture]
    \node[above left=1cm and .8cm of current page.south east] {\includegraphics[width=2.25cm]{#1}};
\end{tikzpicture}}

\title{Introduction to OpenCV}
\date{\today}
\author{Satoshi Murashige}
\institute{Mathematical Informatics Lab., NAIST}

\begin{document}
	\begin{frame}[plain]
		\maketitle
		%\myinsertlogo{spam.pdf}
	\end{frame}
	\begin{frame}{Text books}
		『詳解 OpenCV』
	\end{frame}
	\begin{frame}{Table of Contents}
		\begin{enumerate}
			\item 基本的な画像処理
				\begin{itemize}
					\item 1章:概要
					\item 2章:OpenCV入門
					\item 10章:フィルタとコンボリューション
				\end{itemize}
			\item 物体検出
				\begin{itemize}
					\item 12章:画像解析
					\item 13章:ヒストグラムとテンプレートマッチング
					\item 14章:輪郭
				\end{itemize}
			\item 動画解析
				\begin{itemize}
					\item 15章:背景除去
					\item 16章:キーポイントと記述子
					\item 17章:トラッキング
				\end{itemize}
			\item 3次元復元
				\begin{itemize}
					\item 18章:カメラモデルとキャリブレーション
					\item 19章:射影変換と3次元ビジョン
				\end{itemize}
		\end{enumerate}
	\end{frame}
\end{document}
